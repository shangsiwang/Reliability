\documentclass{article}
\input{preamble.tex}

\title{\vspace{-50pt}
% Reliable and Reproducible Brain-Based Measures for Candidate Biomarker Discovery from Big Data
% Optimal Discovery Science from Big Brain-Imaging Data via Reliability and Reproducibility  
\db{Optimal Design for Discovery Science via Maximizing Discriminability: \\ Applications in Neuroimaging}
}
\author{Shangsi Wang, Zhi Yang, Xi-Nian Zuo, Michael Milham, Cameron Craddock,  \\ 
Greg Kiar, William Gray Roncal, Eric Bridgeford,
Consortium for Reliability and Reproducibility, \\ Carey E.~Priebe, Joshua T. Vogelstein}


\begin{document}

\maketitle
\tableofcontents
\newpage

\section{Introduction}

\para{Opportunity} Benchmark datasets are omnipresent

\para{Challenge} Deciding on how to collect and process the data, in the absense of an explicit task or for multiple tasks.

\para{Action} We define discriminability as a probability which bounds Bayes error.

\para{Resolution} We compute discriminability on a simulated data set and real data sets to study the optimal way to process data.



\section{Results}



\subsection{Theory}

\subsubsection{Discriminability as a framework to guide processing}
\para{Rigorously define discriminability} Based on intuition, we define discriminability of a multi-subject distribution $F$ as a probability of within subject distances to be smaller than across subject distances.

\para{Define optimizing pipeline problem} we are looking for most discriminable processing pipeline that is $\max_\phi \phi(F)$.



\subsubsection{Optimizing discriminability Optimizes Bound on Performance for Any Task}

\para{Introduce classification} Define the binary classification problem and Bayes error.

\para{Justify discriminability} Under additive noise setting, discriminability bounds bayes error.

\begin{thm}
discriminability bounds predictive accuracy
\end{thm}

\para{Corollary} Processing pipeline selection.

\subsubsection{Estimating Discriminability}

\para{Estimate discriminability} We design an estimator Dhat to estimate discriminability from test-retest data set.

\begin{itemize}

\item In a model free setting, $E(Dhat)=D$.

\item In a model free setting, $\mh{D}_n \conv D$
\end{itemize}


\subsection{Simulations}


\subsubsection{Dhat $\conv$ E(D) under gaussian simulation}
\para{Simulation} The means of subjects follow gaussian and the observations from a subject also follow gaussian. 

\begin{figure}[ht!]
	\includegraphics[width=3.0in]{../Figs/simumnr_violin.png}
	\caption{{\bf Convergence of sample discriminability.} Distribution of sample discriminability is estimated. As the number of subjects increases, the sample discriminability converges to true population discriminability. }
	\label{fig:1}
\end{figure}


\subsubsection{we can use Dhat to choose the most discriminable parameter }
\para{Simulation} Learn the optimal linear projection through discriminability

\begin{figure}[ht!]
	\includegraphics[width=3.0in]{../Figs/parameter_selection_2sub.png}
	\caption{{\bf Linear projection based on by PCA and Discriminability.} Linear projections are computed using PCA, optimizing Discriminability. Maximizing discriminability yield separated samples which have Bayes optimal classification accuracy. }
	\label{fig:3}
\end{figure}


\subsection{Connectome Applications}

\subsubsection{optimal Discriminability yields optimal predictive accuracy}
\para{real experiment} Describe the data and threshold experiment. 

\para{real experiment} Emphasize that discriminability selects threshold which is close to optimal for multiple tasks.

\begin{figure}[ht!]
\includegraphics[width=4.0in]{../Figs/hcp_2.png}
\caption{{\bf Optimizing Discriminability yields optimal prediction accuracy for multiple covariates.} HCP100 is used to investigate optimal threshold to convert correlation graphs into binary graphs. The threshold is varied from 0 to 1. For each threshold, the discriminability is computed;  sex, age and a neuro factor are predicted using k-NN. The threshold maximizing discriminability is close to optimal thresholds for predicting three covariates. }
\label{fig:4}
\end{figure}

\subsubsection{Best Pipeline of 64}
\para{real experiment} Describe the 12 data sets and 64 pipelines. 

\para{real experiment} Decide the best among 64 pipelines.

\para{real experiment} Decide the optimal for each decision (atlas, nff vs frf, ant vs fsl, nsc vs scr, gsr vs ngs) using anova test.
\begin{figure}[ht!]
	\includegraphics[width=4.0in]{../Figs/64_pipelines_raw.png}
	\caption{{\bf Discriminability of raw fmri graphs from 12 data sets processed 64 ways.}  Discriminability of BNU1, BNU2, DC1, IACAS, IBATRT, IPCAS, JHNU, NYU1, SWU1, UM, UWM and XHCUMS processed by 64 pipelines are computed. Colors indicate data set and size indicates the size of data set. The black square indicates the mean discriminability across 12 data sets. CFXXG pipeline has the best mean discriminability across data set. }
	\label{fig:6}
\end{figure}

\begin{figure}[ht!]
	\includegraphics[width=4.0in]{../Figs/Differ_violin_mean.png}
	\caption{{ \bf Paired difference in discriminability of decisions.} Difference in discriminability for each decision in pipeline is compared by fixing other decisions and data sets. nff and gsr are statistical significantly better than the alternatives. fsl and nsc are not significantly better than the alternatives.}
	\label{fig:7}
\end{figure}

\subsubsection{raw vs ranks}
\para{real experiment} Describe how to convert raw graphs to rank graphs

\para{real experiment} Decide rank is better, especially when global signal regression is not performed.
\begin{figure}[ht!]
	\includegraphics[width=4.0in]{../Figs/64_pipelines_differ.png}
	\caption{{ \bf Paired difference in discriminability between rank and raw graphs.} Difference in discriminability for rank and raw fmri graphs are computed for 12 data sets processed using 64 pipelines. Rank fmri graphs are more discriminable than raw fmri graphs.}
	\label{fig:7}
\end{figure}


\subsubsection{DTI vs. fMRI}
\para{real experiment} Describe the 4 fmri and dti data sets.

\para{real experiment} After removing outliers, dti is more discriminable than fmri.

\begin{figure}[ht!]
	\includegraphics[width=4.0in]{../Figs/dti_mri_differ.png}
	\caption{{ \bf Paired difference in discriminability between dti and fmri data sets.} Discriminability of DTI and fMRI graphs are computed for BNU1, HNU1, SWU4 and KKI data set. DTI data sets are more discriminable than fMRI data sets.}
	\label{fig:7}
\end{figure}

\subsubsection{DTI atlas/resolution}
\para{real experiment} Describe we process a  dti data data set with rank, raw and log for 15 atlases. 

\para{real experiment} We see a trend that large roi are better. Since discriminability after removing outliers is close to one, more experiments need to be done.

\begin{figure}[ht!]
	\includegraphics[width=4.0in]{../Figs/SWU4_DTI_logroi.png}
	\caption{{ \bf Discriminability of 15 atlases.} Discriminability of SWU4 DTI registered with 15 atlases are computed. Atlases with a larger number of rois tend to be more discriminable.}
	\label{fig:7}
\end{figure}

\section{Discussion}

\para{Summary} We propose a definition of discriminability and apply it to a variety of set ups.

\para{Related Work} I2C2, DISCO and GICC


\para{Next Steps}


% \input{intro}
% \input{simulations}
% \input{flow}
% \input{setting}
% \input{logic}
% \input{main}
% \input{setup}
% \input{gRAICAR}



\appendix






\newpage
\small{
\bibliography{biblio}
\bibliographystyle{IEEEtran}
}


\end{document}
