\documentclass{article}
\input{preamble.tex}

\title{\vspace{-50pt}
% Reliable and Reproducible Brain-Based Measures for Candidate Biomarker Discovery from Big Data
% Optimal Discovery Science from Big Brain-Imaging Data via Reliability and Reproducibility  
\db{Optimal Design for Discovery Science via Maximizing Discriminability: \\ Applications in Neuroimaging}
}
\author{Shangsi Wang, Zhi Yang, Xi-Nian Zuo, Michael Milham, Cameron Craddock,  \\ 
Greg Kiar, William Gray Roncal, Eric Bridgeford,
Consortium for Reliability and Reproducibility, \\ Carey E.~Priebe, Joshua T. Vogelstein}


\begin{document}

\maketitle
\tableofcontents
\newpage

\section{Introduction}

\para{Opportunity} Benchmark datasets are omnipresent

\para{Challenge} Deciding on how to collect and process the data, in the absense of an explicit task

\para{Action} We define ``discrimininability'' as the ability to differentiate between different classes of objects

\para{Resolution} 


\section{Results}



\subsection{Theory}

\subsubsection{Definition of Discriminability}

\subsubsection{Optimizing discriminability Optimizes Bound on Performance for Any Task}


\begin{thm}
discriminability bounds predictive accuracy
\end{thm}


\subsubsection{Estimator/Test Statistic}

\begin{itemize}
\item an estimator of D, called Dhat

\item proof that our estimator is unbiased (in a model free setting), 
$E(Dhat)=D$.

\item proof that our estimator asymptotically converges to truth (in a model free setting), $\mh{D}_n \conv D$
\end{itemize}


\subsection{Simulations}

\subsubsection{Dhat $\conv$ E(D) in practice for some (eg, gaussian) simulation}

\begin{figure}[t!]
\includegraphics[width=3.0in]{../Figs/simumnr_violin.png}
\caption{}
\label{fig:64}
\end{figure}




\subsubsection{Dhat provides a more useful bound than ICC or I2C2 for a variety of simulated settings}

\begin{figure}[t!]
\includegraphics[width=3.0in]{../Figs/Figure1_draft.png}
\caption{}
\label{fig:64}
\end{figure}



\subsubsection{we can use Dhat to choose the most discriminable parameter (eg, threshold)}


\begin{figure}[t!]
\includegraphics[width=3.0in]{../Figs/parameter_selection_2sub.png}
\caption{}
\label{fig:64}
\end{figure}




\subsection{Connectome Applications}

\subsubsection{optimal Discriminability yields optimal predictive accuracy}

\begin{figure}[t!]
\includegraphics[width=3.0in]{../Figs/HCP.png}
\caption{}
\label{fig:64}
\end{figure}


\subsubsection{Best Pipeline of 64}

\begin{figure}[t!]
\includegraphics[width=3.0in]{../Figs/64_pipelines_gg.png}
\caption{}
\label{fig:64}
\end{figure}


\subsubsection{best pipeline = product of marginals}


\begin{figure}[t!]
\includegraphics[width=3.0in]{../Figs/Differ_violin_mean.png}
\caption{}
\label{fig:64}
\end{figure}



\subsubsection{thresholding vs. binning ranks}

\subsubsection{which atlas/resolution}


\subsubsection{DTI vs. fMRI}










\section{Discussion}

\para{Summary}

\para{Related Work}


\para{Next Steps}


% \input{intro}
% \input{simulations}
% \input{flow}
% \input{setting}
% \input{logic}
% \input{main}
% \input{setup}
% \input{gRAICAR}



\appendix






\newpage
\small{
\bibliography{biblio}
\bibliographystyle{IEEEtran}
}


\end{document}
