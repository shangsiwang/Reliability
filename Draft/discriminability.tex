\documentclass{article}
\input{preamble.tex}

\title{\vspace{-50pt}
% Reliable and Reproducible Brain-Based Measures for Candidate Biomarker Discovery from Big Data
% Optimal Discovery Science from Big Brain-Imaging Data via Reliability and Reproducibility  
\db{Optimal Design for Discovery Science via Maximizing Discriminability: \\ Applications in Neuroimaging}
}
\author{Shangsi Wang, Zhi Yang, Xi-Nian Zuo, Michael Milham, Cameron Craddock,  \\ 
Greg Kiar, William Gray Roncal, Eric Bridgeford,
Consortium for Reliability and Reproducibility, \\ Carey E.~Priebe, Joshua T. Vogelstein}


\begin{document}

\maketitle
\tableofcontents
\newpage

\section{Introduction}

\para{Opportunity} Benchmark datasets are omnipresent

\para{Challenge} Deciding on how to collect and process the data, in the absense of an explicit task or for multiple tasks.

\para{Action} We define discriminability as a probability which bounds Bayes error.

\para{Resolution} We compute discriminability on a simulated data set and real data sets to study the optimal way to process data.



\section{Results}



\subsection{Theory}

\subsubsection{Discriminability as a framework to guide processing}
\para{Rigorously define discriminability} Based on intuition, we define discriminability of a multi-subject distribution $F$ as a probability of within subject distances to be smaller than across subject distances.

\para{Define optimizing pipeline problem} we are looking for most discriminable processing pipeline that is $\max_\phi \phi(F)$.



\subsubsection{Optimizing discriminability Optimizes Bound on Performance for Any Task}

\para{Introduce classification} Define the binary classification problem and Bayes error.

\para{Justify discriminability} Under additive noise setting, discriminability bounds bayes error.

\begin{thm}
discriminability bounds predictive accuracy
\end{thm}

\para{Corollary} Processing pipeline selection.

\subsubsection{Estimating Discriminability}

\para{Estimate discriminability} We design an estimator Dhat to estimate discriminability from test-retest data set.

\begin{itemize}

\item In a model free setting, $E(Dhat)=D$.

\item In a model free setting, $\mh{D}_n \conv D$
\end{itemize}


\subsection{Simulations}


\subsubsection{Dhat $\conv$ E(D) in practice for some gaussian simulation}
\para{Simulation} The means of subjects follow gaussian and the observations from a subject also follow gaussian. 



\subsubsection{we can use Dhat to choose the most discriminable parameter }
\para{Simulation} Learn the optimal linear projection through discriminability




\subsection{Connectome Applications}

\subsubsection{optimal Discriminability yields optimal predictive accuracy}
\para{real experiment} Describe the data and threshold experiment. 

\para{real experiment} Emphasize that discriminability selects threshold which is close to optimal for multiple tasks.



\subsubsection{Best Pipeline of 64}
\para{real experiment} Describe the 12 data sets and 64 pipelines. 

\para{real experiment} Decide the best among 64 pipelines.

\para{real experiment} Decide the optimal for each decision (atlas, nff vs frf, ant vs fsl, nsc vs scr, gsr vs ngs) using anova test.




\subsubsection{raw vs ranks}
\para{real experiment} Describe how to convert raw graphs to rank graphs

\para{real experiment} Decide rank is better, especially when global signal regression is not performed.

\subsubsection{DTI vs. fMRI}
\para{real experiment} Describe the 4 fmri and dti data sets.

\para{real experiment} After removing outliers, dti is more discriminable than fmri.

\subsubsection{DTI atlas/resolution}
\para{real experiment} Describe we process a  dti data data set with rank, raw and log for 15 atlases. 

\para{real experiment} We see a trend that large roi are better. Since discriminability after removing outliers is close to one, more experiments need to be done.



\section{Discussion}

\para{Summary} We propose a definition of discriminability and apply it to a variety of set ups.

\para{Related Work} I2C2, DISCO and GICC


\para{Next Steps}


% \input{intro}
% \input{simulations}
% \input{flow}
% \input{setting}
% \input{logic}
% \input{main}
% \input{setup}
% \input{gRAICAR}



\appendix






\newpage
\small{
\bibliography{biblio}
\bibliographystyle{IEEEtran}
}


\end{document}
